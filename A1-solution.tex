\documentclass[a4paper]{article}

\usepackage{fullpage} % Package to use full page
\usepackage{parskip} % Package to tweak paragraph skipping
\usepackage{tikz} % Package for drawing
\usepackage{amsmath}
\usepackage{float}
\usepackage{hyperref}
\usepackage[nodisplayskipstretch]{setspace}
\setstretch{0.5}

\title{MAST30013 - Techniques in Operations Research: Assignment 1}
\author{Shreyash Patodia (767336)}
\date{19/03/2018}

\begin{document}

\maketitle

\section*{Question 1}

\subsection*{1(a):}

Below is a plot of $f(x) = 8e^{1 -x} + 7log(x)$ in the interval $[1, 2]$:
\begin{figure}[!htbp]
\begin{center}
\includegraphics[scale=0.75]{problem-1.png}
\end{center}
\caption{Plot of $f(x) = 8e^{1 -x} + 7log(x)$ in the interval $[1, 2]$:}\label{exampleplot}
\end{figure}

The above plot clearly shows that $f(x)$ is unimodal since it only has one local minima.

\subsection*{1(b):}

In order to perform Fibonacci Search we first need to find the smallest $n$ such that $F_n(\alpha) = (b - a)$. Thus, we need to find an $n$ where $\alpha  < 2\epsilon$ or we need:
\begin{equation}
(b - a)/F_n < 2\epsilon
\end{equation}
\begin{equation}
F_n > (b - a)/2\epsilon
\end{equation}
Setting $b = 2, a = 1$ and $\epsilon = 0.08$ we get:
\begin{equation}
F_n > \dfrac{2 - 1}{2 * 0.08}
\end{equation}
\begin{equation}
F_n > \dfrac{1}{0.16}
\end{equation}
\begin{equation}
F_n > 6.25
\end{equation}

So, $F_n = 8$ and $n = 5$.

Now, we know $f(x)$, we know the interval is $[1, 2]$ and the initial $k = 5$.  We will also use the following formulas for p and q:

\begin{equation}
a = 1
\end{equation}

\begin{equation}
b = 2
\end{equation}

\begin{equation}
p = b - \dfrac{F_{k - 1}}{F{k}}(b - a)
\end{equation}

\begin{equation}
q = a + \dfrac{F_{k - 1}}{F{k}}(b - a)
\end{equation}

After, that we will use $f(p)$ and $f(q)$ to evaluate how to reduce the intervals.

%%%%%%%%%%%%%%%%%%%%%%%
\begin{equation}
k = n = 5
\end{equation}

\begin{equation}
p = 2 - \dfrac{5}{8} * (1) = 2 - \dfrac{5}{8} = 1.375
\end{equation}

\begin{equation}
q = 1 + \dfrac{5}{8} * (1)  = 1 + \dfrac{5}{8} = 1.625
\end{equation}

\begin{equation}
f(p) = 7.7275
\end{equation}   

\begin{equation}
f(q) = 7.6806
\end{equation}

Now, since $f(p) > f(q)$ the new interval will have:

\begin{equation}
a = p = 1.375
\end{equation}

\begin{equation}
b = 2
\end{equation}

\begin{equation}
p = q = 1.625
\end{equation}

%%%%%%%%%%%%%%%%%%%%%%%%%%%%%%%%%%%%


%%%%%%%%%%%%%%%%%%%%%%%
\begin{equation}
k = n = 4
\end{equation}

\begin{equation}
q = 1.375 + \dfrac{3}{5} * (0.625)  = 1.75
\end{equation}

\begin{equation}
f(p) = 7.6806    
\end{equation}   

\begin{equation}
f(q) = 7.6962
\end{equation}

Now, since $f(q) > f(p)$ the new interval will have:

\begin{equation}
a = 1.375
\end{equation}
\begin{equation}
b = q = 1.75
\end{equation}

\begin{equation}
q  = p = 1.625
\end{equation}

%%%%%%%%%%%%%%%%%%%%%%%%%%%%%%%%%%%%

%%%%%%%%%%%%%%%%%%%%%%%
\begin{equation}
k = n = 3
\end{equation}

\begin{equation}
p = 1.75 -  \dfrac{2}{3} * (0.375)  = 1.5
\end{equation}

\begin{equation}
f(p) = 7.6905   
\end{equation}   

\begin{equation}
f(q) =  7.6806
\end{equation}

Now, since $f(p) > f(q)$ the new interval will have:

\begin{equation}
a = p = 1.5
\end{equation}
\begin{equation}
b = 1.75
\end{equation}

\begin{equation}
p  = q = 1.625
\end{equation}

%%%%%%%%%%%%%%%%%%%%%%%%%%%%%%%%%%%%

\begin{equation}
k = 2
\end{equation}

Because $k = 2$ we will have to set q as below:

\begin{equation}
q = a + 2\epsilon = 1.660
\end{equation}

\begin{equation}
f(p) = 7.6806    
\end{equation}

\begin{equation}
f(q) = 7.6825
\end{equation}

Now, since $f(p) < f(q)$ the new interval will have:

\begin{equation}
a = 1.5
\end{equation}
\begin{equation}
b = q = 1.660
\end{equation}

Thus, the final interval found using fibonacci search is $[1.5, 1.66]$ which is withing the tolerance required. 

Thus $x^* = 1.58 \pm 0.08$.

\section*{Question 2}

Let us first try to find a $b$ such that $[0, b]$ contains the minimum. But first we need to check that $f(x)$ is unimodal.
Given:
\begin{equation}
f(x) = (40x+1) log(40x+1)−200x
\end{equation}

Plotting $f(x)$ in the interval $[0, 12]$ we get 

\begin{figure}[H]
\begin{center}
\includegraphics[scale=0.75]{problem-2.png}
\end{center}
\caption{Plot of $f(x) = (40x+1) log(40x+1) - 200x$ in the interval $[0, 12]$:}\label{exampleplot}
\end{figure}

Also, let $f'$ is:
\begin{equation}
f'(x) = 40log(40x + 1) - 160
\end{equation}

Setting $f'(x) = 0$:
\begin{eqnarray}
f'(x) &=& 0 \\
40log(40x + 1) - 160 &=& 0 \\
log(40x + 1) - 4 &=& 0 \\
log(40x + 1) &=& 4 \\ 
x &=& \dfrac{e^4 - 1}{40} \\ 
\end{eqnarray}

This shows there is one value for which $f'(x) = 0$ and from the graph we can see this value to be between 0 and 5. Thus, the given function is unimodal and continuous (from the graph and because there are no points where the function wouldn't be defined in $[0, \infty]$ allowing us to use Golden Section Search. 

Let us try to find a good interval to do our search in. Assuming T (a small increment):
$T = 1$

Set $k = 1$:
\begin{equation}
p = 0
\end{equation}
\begin{equation}
q = T
\end{equation}

\begin{equation}
f(p) = f(0) =  0
\end{equation}
\begin{equation}
f(q) = f(1) = -47.7435
\end{equation}

So, $f(p) > f(q)$. Set $k = 2$ :

\begin{equation}
p = q = 1
\end{equation}
\begin{equation}
q = p + 2^{k - 1}T = 1 + 2*1 = 3
\end{equation}

\begin{equation}
f(p) = f(1) = -47.7435
\end{equation}

\begin{equation}
f(q) = -19.7093
\end{equation}


Thus, $ f(p) < f(q)$, meaning the minimum lies in the range: $[0, 3]$.  As $f(p) > f(q)$ for $k = 1$  and $f(p) < f(q)$ for $k = 2$.

We can now perform the Golden Section Search in the interval $[0, 3]$. Now, let us determine the number of calculations needed ($a = 0$, $b = 3$, $\epsilon = 0.3$):

\begin{equation}
a = 0
\end{equation}
\begin{equation}
b = 3
\end{equation}

\begin{eqnarray}
\gamma^n (b - a) &<& 2\epsilon\\ 
0.6180^n &<& \dfrac{0.6}{3}\\
0.6180^n &<& 0.2 \\
0.6180^4 = 0.1459 &<& 0.2
\end{eqnarray}

Thus, $ n = 4 $ which means we need to do 5 f calculations.

\begin{equation}
p = b - \gamma(b - a) = 3 - 0.6180(3) = 1.146
\end{equation}
\begin{equation}
q = a + \gamma(b - a) = 0 + 0.6180(3) = 1.854
\end{equation}
\begin{equation}
f(p) = -49.0182
\end{equation}
\begin{equation}
f(q) = -46.1361
\end{equation}

Since, $f(q) > f(p)$, we can set the new $b = q = 1.854$ and $a = 0$ remains the same. We have $3$ f calculations left (Also, $q = p = 1.146$):

\begin{equation}
p = 1.854  - 0.6180*(1.854) = 0.7082
\end{equation}
\begin{equation}
f(p) = -42.5542
\end{equation}

\begin{equation}
f(q) = -49.0182
\end{equation}

Since, $f(p) > f(q)$. We get $a = p = 0.7082 $, $p = q = 1.146$ and $b = 1.854$ remains the same. We now have just $2$ f calculations left. 

Now, 

\begin{equation}
q = 0.7082 + 0.6180*(1.1458) = 1.4163
\end{equation}
\begin{equation}
f(p) = -49.0182
\end{equation}

\begin{equation}
f(q) = -49.5141
\end{equation}

Since, $f(p) > f(q)$, we make $a = p = 1.146$, $p = q = 1.4163$ and $b = 1.854$ remains the same. Doing the last f calculation. 

\begin{equation}
q = 1.146 + 0.6180*(0.708) = 1.5834
\end{equation}
\begin{equation}
f(p) = -49.5141
\end{equation}
\begin{equation}
f(q) = -48.7760
\end{equation}

Finally, since $f(q) > f(p)$ then the final $a = 1.146$ and the final $b = q = 1.5834$. Thus, the interval for minimisation is $[1.146, 1.583]$.  Which means $x^* = 1.3645 \pm 0.2185$.

\section*{Question 3}

Given:

$
f(x) = (2x − 1)^3 + 4(4 − 10x)^4
$

$f'(x) = 3(2x - 1)^2 + 16(4 - 10x)^3$

Now, let  $a = 0$ and $T = 1$ such that, 

\begin{equation}
p = a = 0
\end{equation}

\begin{equation}
q = a + T = 1
\end{equation}

\begin{equation}
f(p) = -1 + 1024 = 1023
\end{equation}

\begin{equation}
f(q) = 1 + 5184  = 5185
\end{equation}











\end{document}